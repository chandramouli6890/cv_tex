\documentclass{mycv}

\newboolean{GERMAN_CV}\setboolean{GERMAN_CV}{false}


\newcommand{\CVRole}{senior software engineer}
\newcommand{\CVFields}{autonomous driving, machine learning and artificial intelligence}


\begin{document}
\sloppy % this restricts words spilling out of the margins
\color{templateColor1}
\pagenumbering{gobble}
% \AddToShipoutPicture{\BackgroundPic}

\normalfont
\begin{minipage}[c]{0.32\textwidth}
    %\centering
  %\includegraphics[width=5.2cm]{../img/CV_Photo_comp.png}
  %\includegraphics[width=5.45cm]{../img/IMG_7896_m02.jpg}
  % \includegraphics[width=5.45cm]{../img/CV_Photo_new_comp.png}
  \includegraphics[width=5.7cm]{../img/CV_Photo_Informal_Comp.png}
\end{minipage}
\begin{minipage}[]{0.8\textwidth}
    \vspace{5mm}

    {\huge Dr.-Ing }\\

    {\Huge Chandramouli}\\

    {\Huge Gnanasambandham}
    \vspace{2mm}

    \vspace{2mm}

    {\large
        Steinweg 24\\
        71263 Weil der Stadt\\

        \ifthenelse{\boolean{GERMAN_CV}}
        {
            \dateOfBirthIcon 6. August 1990\\
            \maritalStatusIcon Verheiratet\\
        }
        {
            \dateOfBirthIcon 6\textsuperscript{th} August 1990\\
            \maritalStatusIcon Married, no children\\
        }
        \telephoneIcon +49 179 6588043\\
        \mailIcon \href{mailto:chandramouli681990@gmail.com}{\link{chandramouli681990@gmail.com}}
    }
  
    \vspace{13mm}
\end{minipage}

{\rlap{\color{templateColor4}\rule[0mm]{\textwidth}{\ulinewidth}}}
\columnratio{0.39}
\setlength{\columnsep}{2.5em}
\setlength{\columnseprule}{\ulinewidth}
\colseprulecolor{templateColor4}
\begin{paracol}{2}
    \ifthenelse{\boolean{GERMAN_CV}}
    {
        \lsection{Angestrebte Position}

        Ich bin ein leidenschaftlich neugieriger Ingenieur mit hervorragenden
        interkulturellen Kommunikationsf{\"a}higkeiten. In meiner aktuellen
        Position leite ich die Entwicklung robuster Fahrzeugmodelle für
        hochskalierbare Simulationen, die von über 200 aktiven Nutzern genutzt
        werden. Darüber hinaus war ich w{\"a}hrend meiner Zeit an der
        Universit{\"a}t Erstautor von 6 Artikeln in renommierten
        Fachzeitschriften im Bereich Partikeldynamik. All dies wurde durch
        meine au{\ss}ergewöhnliche Anpassungsf{\"a}higkeit sowie meine herausragenden
        analytischen und Teamf{\"a}higkeiten erm{\"o}glicht. Jetzt strebe ich eine neue
        Herausforderung als Senior Engineer an, um meine Expertise in
        Simulation in innovative Mobilit{\"a}tsl{\"o}sungen einzubringen.
    } 
    { 

        \lsection{Profile}

        I am a passionately curious engineer with excellent
        intercultural communication skills. I have been spearheading the
        development of robust multi-fidelity vehicle models for highly-scalable
        simulations with over 300 active users in my current organization.
        Moreover, I have been first author of 6 peer-reviewed journal articles in
        the field of particle dynamics during my time at the academia. All this was
        possible, thanks to my exceptional adaptability to rapidly changing
        environments and my extraordinary analytical and team
        skills. I am now seeking new opportunities as a Senior Engineer,
        where I can leverage my expertise in simulation and analytical
        problem-solving to contribute to cutting-edge innovations in mobility.\\
    }

  \ifthenelse{\boolean{GERMAN_CV}}
  {
      \lsection{Sprachen}
      \begin{onehalfspace}
      \begin{tabular}{
                p{2cm} >{\raggedleft\arraybackslash}p{4.5cm}}
                {\mybox\mybox\mybox\mybox\mybox}  & {flie{\ss}end| Deutsch} \\
                {\mybox\mybox\mybox\mybox\mybox} & {flie{\ss}end | Englisch}\\
                {\mybox\mybox\mybox\mybox\mybox}  & {Muttersprache | Tamil}  \\
                {\mybox\mybox\mybox\mybox\myboxo}  & {fortgeschritten | Hindi}\\\\
      \end{tabular}
      \end{onehalfspace}
  }
  {

      \lsection{Languages}
      \begin{onehalfspace}
          \begin{tabular}{
                  p{2cm} >{\raggedleft\arraybackslash}p{4.5cm}}
                  {\mybox\mybox\mybox\mybox\mybox}  & {Proficient | German} \\
                  {\mybox\mybox\mybox\mybox\mybox} & {Proficient | English}\\
                  {\mybox\mybox\mybox\mybox\mybox}  & {Mother tounge | Tamil}  \\
                  {\mybox\mybox\mybox\mybox\myboxo}  & {Advanced | Hindi}\\\\
          \end{tabular}
      \end{onehalfspace}
  }

        \lsection{Web}
        \begin{minipage}[c]{0.31\textwidth}
            \begin{flushright}
                {\bfseries Linkedin}\\
                {\footnotesize
                    \href{https://linkedin.com/in/gnanasambandhamc}{\link{linkedin.com/in/gnanasambandhamc}}}
            \end{flushright}
        \end{minipage}
        \begin{minipage}{0.05\textwidth}
            \linkedinIcon
        \end{minipage}
        \vspace{3mm}

        \begin{minipage}[c]{0.31\textwidth}
            \begin{flushright}
                {\bfseries Medium}\\
                {\footnotesize \href{https://chandramoulig.medium.com}{\link{chandramoulig.medium.com}}}
            \end{flushright}
        \end{minipage}
        \begin{minipage}{0.05\textwidth}
            \mediumIcon
        \end{minipage}
        \vspace{3mm}

        \begin{minipage}[c]{0.31\textwidth}
            \begin{flushright}
                {\bfseries Matlab}\\
                {\footnotesize
                    \href{https://de.mathworks.com/matlabcentral/profile/authors/4267772}{\link{MatlabCentral    Profile}}}
            \end{flushright}
        \end{minipage}
        \begin{minipage}{0.05\textwidth}
            \matlabIcon
        \end{minipage}

\switchcolumn
\ifthenelse{\boolean{GERMAN_CV}}
{
    \rsection{Ausbildung}
        \subsection{May 2016 - April 2021}{Promotion zum
        Dr.-Ing.}{Universit{\"a}t Stuttgart}, {Abschlussnote: 1.0 ({\bfseries Magna cum
        laude})}
          \begin{itemize}
              \item Dissertationstitel: Particle Dampers- Enhancing
                  Energy Dissipation using Fluid/Solid Interactions and Rigid
                  Obstacle-Grids
          \end{itemize}
        
        \subsection{Oktober 2012 - April 2016}{Master of Science in Commercial
        Vehicle Technology}{Technische Universit{\"a}t Kaiserslautern,
        {Abschlussnote: 1.9}}\\
        
        \subsection{Juni 2008 - April 2012}{Bachelor of Engineering in
        Fertigunstechnik} {Anna University, Chennai, Indien, {Abschlussnote:
        8.3/10} ({\bfseries sehr gut})}\\

}
{
    \rsection{Educational Qualification}
        \subsection{May 2016 - April 2021}{Ph.D. in Mechanical Engineering}
        {University of Stuttgart}
          \begin{itemize}
              \item Dissertation Titel: Particle Dampers- Enhancing
                  Energy Dissipation using Fluid/Solid Interactions and Rigid
                  Obstacle-Grids
          \end{itemize}

        \subsection{October 2012 - April 2016}{Master of Science in Commercial Vehicle
        Technology}{Technical University of Kaiserslautern, {Grade: 1.9}}\\

        \subsection{June 2008 - April 2012}{Bachelor of Engineering in
            Production Engineering}{Anna University, Chennai, India, {Grade CGPA: 8.3/10}}\\
}

\ifthenelse{\boolean{GERMAN_CV}}
{
    \rsection{Beruflicher Werdegang}
    \subsection{April 2023 - heute}{Torc Europe GmbH, Stuttgart}{Staff software engineer}
          \begin{itemize}

            \item Leitung eines Teams zur Entwicklung eines skalierbaren
                Fahrzeugmodells in C++ mit Test-Driven Development (TDD) und
                objektorientierter Programmierung (OOP).

            \item Integration von Fahrzeugmodellen in einen ROS-basierten
                Simulator zur virtuellen Validierung von Level-4 autonome LKWs. 

            \item Zusammenarbeit mit externen Partnern, um eine skalierbare
                Qualifizierungsstrategie für Fahrzeugmodelle gem{\"a}{\ss} den
                ISO-26262 zu entwickeln.
          \end{itemize}
     
          \subsection{August 2021 - M{\"a}rz 2023}{Daimler Truck AG, Stuttgart}{Entwicklungsingenieur, Autonomous Technology Group}
           \begin{itemize}
               \item Entwicklung Fahrzeugmodelle mit unterschiedlichem
                   Detailierungsgrad f{\"u}r skalierbare Simulationen in MATLAB/Simulink.

               \item Entwicklung einer C++ Co-Simulations-Schnittstelle zur
                   Kopplung eines hochdetaillierten Mehrk{\"o}rpermodells mit
                   einem virtuellen Fahrer für hochdynamischen
                   Man{\"o}versimulationen.

           \end{itemize}
}
{
    \rsection{Professional Career}
    \subsection{April 2023 - present}{Torc Europe GmbH, Stuttgart}{Staff software engineer}
          \begin{itemize}
            \item Spearheaded a team develop a highly-scalable
                vehicle model in C++ following Test-Driven Development
                (TDD) and Object-Oriented Programming (OOP)
            \item Integrated vehicle models into a Robotic Operating System (ROS)
                based simulator to enable virtual validation of Level 4
                autonomous vehicles
            \item 
                Collaborated with external stakeholders to develop a scalable
                validation strategy for vehicle models as per
                ISO-26262.


          \end{itemize}
    
    \subsection{August 2021 - March 2023}{Daimler Truck AG, Stuttgart}{Vehicle model engineer}
          \begin{itemize}
              \item Develop multi-fidelity vehicle models for scalable simulations in the context of
          virtual validation in MATLAB/Simulink

            \item Developed a co-simulation interface in C++ to couple a high-fidelity
                multibody truck model and the virtual driver using
                TCP/IP communication interface.

          \end{itemize}
}
\end{paracol}

{\rlap{\color{templateColor4}\rule[0mm]{\textwidth}{\ulinewidth}}}
\begin{paracol}{2}
    \switchcolumn
    \ifthenelse{\boolean{GERMAN_CV}}
    {
        \rsection{Beruflicher Werdegang (Fortsetzung)}
        \subsection{Mai 2016 - April 2021}{Universit{\"a}t
        Stuttgart}{Wissenschaftlicher Mitarbeiter am Institut f{\"u}r
        technische und numerische Mechanik (ITM)}\\

        \subsection{October 2015 - April 2016}{Fraunhofer Institute (ITWM), Kaiserslautern}
            {Werkstudent in der Abteilung Mathematik für die Fahrzeugentwicklung}\\
    }
    {
        \rsection{Professional Career (continued)}
        \subsection{May 2016 - April 2021}{University of Stuttgart}{Scientific staff at the institute for engineering and
            computational \quad\quad mechanics (ITM)}
        \subsection{October 2015 - April 2016}{Fraunhofer Institute (ITWM), Kaiserslautern}
            {Intern in the department of mathematics for vehicle engineering}\\
    }
    
    \ifthenelse{\boolean{GERMAN_CV}}
    {
        \rsection{Technische Qualifikationen}
        \begin{doublespace}
            \begin{tabular}{p{5cm}!{\color{templateColor1}\vrule}p{6.5cm}}
            {\bfseries Programmiersprachen: } & {\bfseries Betriebssystem:}\\
            {\mybox\mybox\mybox\mybox\mybox 9 Jahre | C/C++}  &
            {\mybox\mybox\mybox\mybox\mybox Linux (Debian, Ubuntu)}\\
            {\mybox\mybox\mybox\mybox\mybox 9 Jahre | MATLAB} & 
            {\mybox\mybox\mybox\mybox\myboxo Microsoft Windows}\\
            {\mybox\mybox\mybox\mybox\myboxo 9 Jahre | BASH}  & \\
            {\mybox\mybox\mybox\myboxo\myboxo 6 Jahre | Python}  & \\
        \end{tabular}\vspace{4mm}
        \end{doublespace}
    }
    {
        \rsection{Technical Skills}
        \begin{doublespace}
            \begin{tabular}{p{5cm}!{\color{templateColor1}\vrule}p{6.5cm}}
            {\bfseries Programming Languages: } & {\bfseries Operating System:}\\
            {\mybox\mybox\mybox\mybox\mybox 12 years | C/C++}  &
            {\mybox\mybox\mybox\mybox\mybox Linux (Debian, Ubuntu)}\\
            {\mybox\mybox\mybox\mybox\mybox 12 years | MATLAB} & 
            {\mybox\mybox\mybox\mybox\myboxo Microsoft Windows}\\
            {\mybox\mybox\mybox\mybox\myboxo 9 years | BASH}  & \\
            {\mybox\mybox\mybox\myboxo\myboxo 6 years | Python}  & \\
        \end{tabular}\vspace{4mm}
        \end{doublespace}
    }

    \ifthenelse{\boolean{GERMAN_CV}}
    {
        {\bfseries Prgramm-Kenntnisse:}
        \begin{itemize}
            \item {\bfseries MATLAB/Simulink:} Modellierung, Simulation,
                Optimierung, SiL/DiL simulations, MATLAB GUI, FMI
            \item {\bfseries C/C++:} MEX API, SilverBypass, FMI, ROS, TCP/IP and UDP
            \item{\bfseries Mehrk{\"o}rpersimulation:}  LMS Virtual.Lab Motion, Neweul-M$^2$, 
                MSC Adams, Project Chrono
            \item{\bfseries ADAS/AD-Simulationen:} Applied Object-Sim, IPG CarMaker, IPG TruckMaker
            \item {\bfseries sonstige Programme:}  Silver Virtual-ECU, COMSOL
                Multiphysics, OptiSlang, Oracle VM VirtualBox
        \end{itemize} \par

        {\bfseries Software Entwicklung:}\par
        \begin{itemize}
            \item {\bfseries CI Tools:} Git, Github CLI, Jenkins, Docker\par
            \item {\bfseries Technologien:} PETSc, EIGEN, OpenGL\par
            \item {\bfseries Debuggers/Profilers:} {\verb|gdb}, {\verb|valgrind}, {\verb|calgrind},
                Intel VTune\\
        \end{itemize}
    }
    {
        {\bfseries Simulation and Data Skills:}
        \begin{itemize}
            \item {\bfseries MATLAB/Simulink:} Modelling, \,simulation,
                numerical optimization, SiL/DiL simulations, MATLAB GUI, FMI
            \item {\bfseries C/C++:} MEX API, SilverBypass, FMI, ROS, TCP/IP and UDP
            \item{\bfseries Multibody-Simulation:}  LMS Virtual.Lab Motion, Neweul-M$^2$, 
                MSC Adams, Project Chrono
            \item{\bfseries ADAS/AD-Simulation Tools:}  Applied Object-Sim, IPG CarMaker, IPG TruckMaker
            \item{\bfseries Multibody-Simulation:}  LMS Virtual.Lab Motion, Neweul-M$^2$
            %\item {\bfseries Partikelsimulation:} Pasimodo, Project Chrono, DualSPHysics
            %\item {\bfseries Data Visualization:} Paraview, PlotlyDash, Matplotlib, MATLAB 
            \item {\bfseries Other Software Tools:}  Silver Virtual-ECU, COMSOL
                Multiphysics, OptiSlang, Oracle VM VirtualBox
            %\item {\bfseries Sonstiges:} \LaTeX{}, TikZ, Inkscape, MS Office
        \end{itemize} \par

        {\bfseries Software Development Tools:}\par
        \begin{itemize}
            \item {\bfseries CI Tools:} Git, Github CLI, Jenkins, Docker\par
            \item {\bfseries Technologies:} PETSc, EIGEN, OpenGL\par
            \item {\bfseries Debuggers/Profilers:} {\verb|gdb}, {\verb|valgrind}, {\verb|calgrind},
                Intel VTune\\
        \end{itemize}
    }

\switchcolumn
\ifthenelse{\boolean{GERMAN_CV}}
{
    \lsection{Preise}
    {\RaggedLeft \bfseries Best Presentation Award 2014\\}
    Title: Optimization of Vehicle Parameters based on Lap-Time
    Simulations using Multiobjective Evolutionary Algorithm\\

    {\RaggedLeft \bfseries Best Presentation Award 2015\\} Title: An Adaptive
    Approach to Real-Time Estimation of Vehicle Dynamics Parameters using
    Kalman Filtering\\\\ {\footnotesize Der Preis wurde von der Firmal ALTEN
    GmbH gestiftet und war mit {\bfseries500\,\euro{}} dotiert.}\\

    \lsection{Sonstige Projekte} 

    {\RaggedLeft July 2020 - heute\\ \bfseries Raspberry Pi gesteuerte
    Smart-Home\\} Im Rahmen eines laufenden Hobbyprojekts habe ich ein
    vielseitiges Raspberry-Pi-Smart-Home-Netzwerk aufgebaut. Es umfasst
    Remote-SSH-Zugriff, einen flexiblen Datenserver mit
    automatischen Backups {\"u}ber {\verb|rsync}, einen Zigbee2Mqtt-Server zur
    Steuerung von IoT-Geräten z.B. über Siri.\\

    {\RaggedLeft Juni 2015\\ \bfseries Machine Learning Suite\\} 
    Implementierung eines Deep-Convolution-Neural-Networks zur optischen
    Zeichenerkennung im Rahmen eines freiberuflichen Softwareprojekts in
    MATLAB. Zur Leistungssteigerung wurde die {MEX-API} genutzt. \\

    {\RaggedLeft Juni 2014\\ \bfseries Driver-in-the-Loop Simulator\\} 
    Im Rahmen meiner Arbeit für ein Formula-Student-Rennteam entwickelte ich
    einen Driver-in-the-Loop-Simulator auf Basis einer
    Kommunikationsschnittstelle zwischen IPG CarMaker und MATLAB/Simulink.\\
}
{ 
    \lsection{Awards}
    {\RaggedLeft \bfseries Best Presentation Award 2014\\}
    Optimization of Vehicle Parameters based on Lap-Time
    Simulations using Multiobjective Evolutionary Algorithm\\

    {\RaggedLeft \bfseries Best Presentation Award 2015\\} An Adaptive Approach
    to Real-Time Estimation of Vehicle Dynamics Parameters using Kalman
    Filtering\\\\ 
    {\footnotesize Both awards were offered by ALTEN GmbH,
    complemented with a cash-prize of {\bfseries500\,\euro{}} respectively.}\\

    \lsection{Other Fun Projects} 

    {\RaggedLeft July 2020 - present\\ \bfseries
    Raspberry Pi Powered Smart-Home\\} As part of a on-going hobby project, I
    have built a versatile Raspberry-Pi smart home network with
    remote-ssh-access, custom file storage server with automatic backups using
    {\verb|rsync}, Zigbee2Mqtt server for controlling IOT devices using
    siri/google-nest and custom automations.\\

    {\RaggedLeft Juni 2015\\ \bfseries Machine Learning Suite\\} Implementation
    of a deep convolution neural network for optical character recognition as
    part of a freelance software project in MATLAB. To increase performance the
    {MEX API} was used.\\

    {\RaggedLeft Juni 2014\\ \bfseries Driver-in-the-Loop Simulator\\} As part
    of my work for a formula student racing team, I developed a
    driver-in-the-loop simulator based on a communication interface between
    {IPG CarMaker} and {MATLAB/Simulink}.\\
}

\switchcolumn
\ifthenelse{\boolean{GERMAN_CV}}
{
    \rsection{Ausgew{\"a}hlte Publikationen}
}
{
    \rsection{Selected Publications}
}
    {\footnotesize
    {\bfseries Gnanasambandham}, C.; Fleissner, F.; Eberhard, P.: Enhancing the
    Dissipative Properties of PDs using Rigid Obstacle-Grids. 
    Journal of Sound and Vibration, Vol. 484, p. 115522, 2020.\\
    {\bfseries Gnanasambandham}, C.; Stender, M.; Hoffmann, N.; Eberhard, P.:
    Multi-Scale Dynamics of PDs using Wavelets: Extracting Particle
    Activity Metrics from Ring Down Experiments. Journal of Sound Vibration,
    Vol. 454, pp. 1-13, 2019.\\
    % {\bfseries Gnanasambandham}, C.; Sch{\"onle}, A.; Eberhard, P.: Investigating
    % the Dissipative Effects of Liquid Filled PDs using Coupled DEM-SPH
    % Methods. Computational Particle Mechanic, Vol. 6, pp. 257-169, 2019.\\
    % }
}
\end{paracol}

\begin{figure}[h]
    \begin{picture}(100,50)
        \put(370,0){\includegraphics[width=5.0cm]{../img/Gnanasambandham_Signature.png}}
    \end{picture}
\end{figure}
\ifthenelse{\boolean{GERMAN_CV}}
{
    \vspace{-0.7cm}\hspace{5.5cm} Stuttgart, den 25. November 2024 \quad \hrulefill\\
}
{
    \vspace{-0.7cm}\hspace{5.5cm} Stuttgart, \today \quad \hrulefill\\
}
\raggedleft Dr.-Ing Chandramouli Gnanasambandham

\end{document} 
